\section{Introduction}
\label{sec:introduction}

Natural language interfaces have significantly transformed how users interact with information systems. In educational contexts, students frequently need quick access to academic data, such as course offerings, exam schedules, professor contacts, and enrollment status, which is typically scattered across multiple university systems. Traditional navigation through web portals often requires familiarity with their structure and can be time-consuming, especially for simple queries that could be more efficiently handled through direct conversation.

This project presents a conversational AI agent for the Facultat d'Informàtica de Barcelona (FIB) at Universitat Politècnica de Catalunya (UPC). The agent leverages large language models (LLMs) and the official FIB API to provide natural language access to university information, enabling students to ask questions in plain language rather than navigating complex interfaces.

\subsection{Problem Statement}

Students at FIB face several challenges when seeking academic information. First, academic data is often fragmented across multiple systems, including the FIB API, Racó (the student portal), and the official website, each featuring different interfaces and authentication requirements. Second, navigation complexity can be a barrier, as finding specific information like exam rooms, professor emails, or course prerequisites often requires multiple interactions and prior knowledge of the information architecture. Furthermore, student queries frequently contain implicit context, such as asking "When is my next exam?" which traditional search engines cannot interpret without access to the user's specific enrollment data. Finally, the need for real-time information is critical, particularly during exam periods when immediate access to schedule changes or announcements is essential. These challenges highlight the need for an intelligent interface capable of understanding natural language, integrating diverse data sources, and personalizing responses based on user context.

\subsection{Objectives}

The primary objectives of this project center on building a comprehensive conversational agent. We aim to develop a system that understands natural language queries regarding FIB academic information, including courses, exams, professors, schedules, and news. A key technical goal is to integrate with the FIB API to access both public endpoints, such as the course catalog and faculty directory, and private OAuth-protected endpoints that provide user-specific data like enrolled courses and personal schedules. To manage this complexity, we intend to implement a hierarchical agent architecture that delegates specialized queries to subagents, allowing for modular tool organization and focused system prompts. Additionally, we seek to create a rigorous evaluation framework using the LLM-as-judge methodology to systematically assess response quality across dimensions such as relevance, helpfulness, and appropriate tool usage. Finally, we aim to provide tool interoperability through a Model Context Protocol (MCP) server, thereby enabling other AI assistants to access and utilize FIB API tools.

\subsection{Scope}

This project focuses specifically on information retrieval from the FIB API. The system capabilities include course catalog search with detailed information, exam schedule queries by course or date range, professor and faculty search, and retrieval of FIB news and classroom listings. Crucially, the agent supports user-specific data access via OAuth, allowing students to query their profiles, enrolled courses, personal schedules, and notices. For information not available within the FIB API, the system includes an external web search fallback. Conversely, certain functionalities are explicitly excluded from the current scope. These include administrative actions such as course enrollment or grade submission, multi-turn conversation memory where queries maintain state across interactions, and integration with non-FIB university systems.

\subsection{Document Structure}

The remainder of this report is organized to provide a comprehensive overview of the system's design and performance. \secref{sec:methodology} describes the theoretical background on LLM agents, the hierarchical design approach, and the tools and technologies employed. \secref{sec:implementation} details the technical implementation, including the system architecture, core components, and specific engineering challenges. \secref{sec:results} presents the evaluation framework, quantitative metrics, and a qualitative analysis of the agent's performance. Finally, \secref{sec:conclusions} summarizes the contributions, discusses the current limitations, and outlines potential directions for future work.
